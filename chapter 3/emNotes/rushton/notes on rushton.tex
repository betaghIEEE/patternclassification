%
%  notes on rushton
%
%  Created by  on 2007-04-11.
%  Copyright (c) 2007 Texas Tech University. All rights reserved.
%
\documentclass[]{article}

% Use utf-8 encoding for foreign characters
\usepackage[utf8]{inputenc}

% Setup for fullpage use
\usepackage{fullpage}

% Uncomment some of the following if you use the features
%
% Running Headers and footers
%\usepackage{fancyhdr}

% Multipart figures
%\usepackage{subfigure}

% More symbols
%\usepackage{amsmath}
%\usepackage{amssymb}
%\usepackage{latexsym}

% Surround parts of graphics with box
\usepackage{boxedminipage}

% Package for including code in the document
\usepackage{listings}

% If you want to generate a toc for each chapter (use with book)
\usepackage{minitoc}

% This is now the recommended way for checking for PDFLaTeX:
\usepackage{ifpdf}

%\newif\ifpdf
%\ifx\pdfoutput\undefined
%\pdffalse % we are not running PDFLaTeX
%\else
%\pdfoutput=1 % we are running PDFLaTeX
%\pdftrue
%\fi

\ifpdf
\usepackage[pdftex]{graphicx}
\else
\usepackage{graphicx}
\fi
\title{A LaTeX Article}
\author{  }

\date{2007-04-11}

\begin{document}

\ifpdf
\DeclareGraphicsExtensions{.pdf, .jpg, .tif}
\else
\DeclareGraphicsExtensions{.eps, .jpg}
\fi

\maketitle


\begin{abstract}
\end{abstract}

\section{Introduction}

Part of the point of the EM algorithm is determining $\vec{y}$ in cases where not all of the parameters are known.  An assumption can be made that the parameters come from a Gaussian distribution in an effort to simplify the solution.   One presumption that may be applied is that $\theta_1 ^* , \theta_2 ^*, ..., \theta_{T-1}^* $ for respective values of $\theta_1, ..., \theta_{T-1}$.  

Two assumptions are made in the formulation of the EM algorithm: 
\begin{itemize}
	\item There is a hypothetical parameter sequence $\hat{\Theta}$ of length $T$ on $\{ 1, ..., N\}$ whose first $T-1$ terms are $\{\theta_1 ^*, \theta_2 ^*, ..., \theta_{T-1} ^* \}$
	\item 
	\[
	\mathcal{L}( \hat{\Theta}) := E [ \frac{dP( \hat{\Theta})}{ dP } | \mathcal{Z}_T ]
	\]
\end{itemize}
Since $\Theta$ consists of unknown parameters, ML can not be computed.  
\[
	Q ( \tilde{\Theta} , \hat{\Theta}) := E_{\hat{\Theta}}[ \log (\frac{dP(\tilde{\Theta})}{ dP(\hat{\Theta})  } ) | \mathcal{Z}_k]
\]

\begin{enumerate}
	\item Input $z_T$
	\item Set $K=0$
	\item Find an initial estimate $\tilde{\theta}^0$ for $\Theta_T$
	\item Set $\hat{\Theta} = \{\theta_1 ^* , \theta_2^*, ..., \theta_{T-1}^*, \hat{\theta}^k  \}$
	\item Using 
	\begin{eqnarray*}
		\hat{x}_k := E[ x_k | Z_k] \\
		\tilde{x}_k := E [ x_k | Z_{k-1}] \\
		\hat{P}_k := cov [x_k | Z_k] \\
		\tilde{P}_k := cov [x_k | Z_{k-1}]
	\end{eqnarray*}
	
\end{enumerate}


\bibliography{}
\end{document}
