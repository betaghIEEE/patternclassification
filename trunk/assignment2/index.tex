%
%  index
%
%  Created by  on 2007-05-10.
%  Copyright (c) 2007 Texas Tech University. All rights reserved.
%
%\documentclass[twocolumn]{article}
\documentclass[12pt ]{article}
% Use utf-8 encoding for foreign characters
\usepackage[utf8]{inputenc}

% Setup for fullpage use
\usepackage{fullpage}

% Uncomment some of the following if you use the features
%
% Running Headers and footers
%\usepackage{fancyhdr}

% Multipart figures
%\usepackage{subfigure}

% More symbols
\usepackage{amsmath}
\usepackage{amssymb}
\usepackage{amsthm}
\usepackage{algorithm,algorithmic}
\usepackage{doublespace}
\usepackage{graphicx}
%\usepackage{latexsym}
\newtheorem{thm}{Theorem}[section]
\newtheorem{adef}[thm]{Definition}
\newtheorem{cor}[thm]{Corollary}
\newtheorem{lem}[thm]{Lemma}
%\theoremstyle{definition}


% Surround parts of graphics with box
\usepackage{boxedminipage}

% Package for including code in the document
\usepackage{listings}

% If you want to generate a toc for each chapter (use with book)
\usepackage{minitoc}

% This is now the recommended way for checking for PDFLaTeX:
%\usepackage{ifpdf}

%\newif\ifpdf
%\ifx\pdfoutput\undefined
%\pdffalse % we are not running PDFLaTeX
%\else
%\pdfoutput=1 % we are running PDFLaTeX
%\pdftrue
%\fi

\title{An Independent Component Analysis experiment using Distributed Computing Group Renaissance.}
\author{ Daniel Beatty }

\date{2007-05-10}

\begin{document}

\maketitle


\begin{abstract}
\end{abstract}

\section{Introduction} % (fold)
\label{sec:introduction}

The original assignment is 
\begin{quote}
	This project is to illustrate how a modified Independent Component Analysis (ICA) can be used to restore an image corrupted with additive Gaussian noise.    

	Choose any natural image (of a reasonable size), add 30\% Gaussian noise to the image, assuming a noisy image model:
	\begin{equation}
	\vec{z} = \vec{x} + \vec{n}
	\end{equation}
	%where include picture  \\
%	``http://www.cis.hut.fi/aapo/papers/IJCN99\_tutorial/img163.gif'', mergeFormatInet is Gaussian, and include picture \\ ``http://www.cis.hut.fi/aapo/papers/IJCNN99\_tutorialweb/img8.gif'' where mergeFormatInet is non-Gaussian.  
	
	Denoise the noisy image by using the Sparse Code Shrinkage Method (closely related to ICA) as described in the following in the following reference \cite{hyvarinen99sparse}. 
	
	Compare the restored image using the above method with standard noise removal filters such the simple median filter or the Wiener filter. 
	
	MATLAB codes for various versions of ICA are available from the home page of A Hyv$\ddot{a}$rien and the Laboratory of Computer and Information Science at the University of Helsinki. 
\end{quote}


% section introduction (end)

\section{Background} % (fold)
\label{sec:background}

There is a version of the Fast ICA libraries available for the free version of MATLAB, Octave, ported by by Daniel Ryan \footnote{High Energy Physics, Tufts University, Boston, MA.}   Likewise, I made version of the Fast ICA library for the Cocoa Renaissance frameworks.  These frameworks and libraries are based on the algorithms found in \cite{appo-ica-book}, which also has the best explanation of the Independent Component Analysis and its concepts.   

As described in the instruction there is one approach for computing out the ICA via the Sparse Code Shrinkage Method.  Images also have characteristic equations, such as in equation \ref{image_character}.  The goal of Sparse Coding Shrinkage is to estimate $a_i$ and $s_i$ by forcing the spareness constraint.  If the image values are treated as the $\vec{x}$ in ICA and $a_i$ and $s_i$ are their ICA equivalents, then ICA can estimate this characteristic equation. 

\begin{eqnarray}
%P \Rightarrow M \times N \\ 
I(x,y) = \sum _{i =1 }^n  a_i (x,y) s_i \label{image_character}%\\
%\mathbf{X} = \mathbf{A}\mathbf{S}
\end{eqnarray}

Sparse Coding Shrinkage represents basis vectors such that only a small number of basis vectors are activated at the same time.  \cite[397]{appo-ica-book}.  It is good for compression and de-noising.   In compression, sparse coding shrinkage endeavors to find the rare samples and allow them to be coded special from other more mundane samples.   In de-noising, sparse coding shrinkage provides the independent components, and a selection of interest determines the ``really active'' components.  It is the de-noising part that the interest of this report. 

From all accounts of sparse coding shrinkage it is an application of ICA such that the local mean component is removed and the local variance is normalized.    Furthermore, PCA is applied to the data.   Sparse coding shrinkage does not specify that the procedure be restricted on how the data is arranged to obtain the mixing matrix.  

\cite[391-400]{appo-ica-book} used a neighborhood patch scheme, sampling $l \times l$ sections an arranging their values as one vector.  The complete set of samples compose the source matrix, and the mixing matrix with their independent components may be estimated.  Some experiments used a sliding neighborhood window as opposed to separating the matrix into patches.

% section background (end)



\bibliography{../patternNotes.bib}
\bibliographystyle{abbrv}
\end{document}
