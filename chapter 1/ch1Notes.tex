adaptive pattern recognition
v/s 
pattern recognition

Pattern Recognition Systems:  Typical


The thing that makes a pattern recognition adaptive is the ability to apply such a methods to many classes.  

Lippman paper on Neural Network 

Why classifier neural network.  It has properties like many declarative inference engine.  

For books on syllabus (enter to BibTeX), Cartment  (Adaptive Resonance Theory).  Reference Fuzzy-Set theory.  


Bessdek 1976 showed application of fuzzy-k-means 


k-means (classifier)

Note on ART methods 

The algorithm has an advantage on unspecified number of clusters.  

\begin{itemize}
	\item Machine perception, pattern recognition systems, learning and adaptation, adaptive pattern recognition.
	\item Statistical pattern recognition: Bayesian decision theory, maximum likelihood and Bayesian estimation, parametric and nonparametric techniques, linear discriminant functions.  
	\item Unsupervised learning and clustering
	\item Neural Networking and adaptive pattern recognition, linear network structure, and mathematical representation, feed forward network and back propagation, the Hopfield approach, self-organizing networks and unsupervised learning.  
\end{itemize}

Several project including a final project will be assigned. 


Reference Yang Dissertation 


Scalar Quantization

Vector Quantization (reference Shannon)



\section{Introduction to Pattern Recognition}
\begin{itemize}
	\item Machine Perception
	\item Pattern Recognition Systems
	\item Design Cycle
	\item Learning and Adaption
\end{itemize}


\section{Design Cycle}

Duda and Hart considers the pattern recognition system loop to consist of
\begin{enumerate}
	\item Collection of data
	\item Choose Features
	\item Choose model 
	\item Train classifier
	\item Evaluate Classifier
\end{enumerate}

\section{Learning and Adaption}

Many problems are considered hard enough such that it is impossible to determine the best classification decision ahead of time.    

Training/ Learning in these cases refers to reducing the error of classification on a specific set of data.    Three basic types are acknowledged in the pattern recognition world:
\begin{itemize}
	\item Supervised learning: Involves a trainer (teacher) to identify categories, and costs for each pattern in the training set.  
	\item Unsupervised learning:  aka clustering, there is no explicit trainer, and the system performs ``natural'' groupings.   
	\item Reinforcement learning:  ``no desired category signal is given, instead, the only teaching feedback is that the tenative category is right or wrong. ''
\end{itemize}

